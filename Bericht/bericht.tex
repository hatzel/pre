\documentclass[
	12pt,
	a4paper,
	BCOR10mm,
	%chapterprefix,
	DIV14,
	listof=totoc,
	bibliography=totoc,
	headsepline
]{scrreprt}

\usepackage[T1]{fontenc}
\usepackage[utf8]{inputenc}
\usepackage[english]{babel}

\usepackage{lmodern}

\usepackage[footnote]{acronym}
\usepackage[page,toc]{appendix}
\usepackage{fancyhdr}
\usepackage{float}
\usepackage{graphicx}
\usepackage[pdfborder={0 0 0}]{hyperref}
\usepackage[htt]{hyphenat}
\usepackage{listings}
\usepackage{lscape}
\usepackage{microtype}
\usepackage{nicefrac}
\usepackage{subfig}
\usepackage{textcomp}
\usepackage[subfigure,titles]{tocloft}
\usepackage{units}
\usepackage{pgf}
\usepackage{amsmath}

\lstset{
	basicstyle=\ttfamily,
	frame=single,
	numbers=left,
	language=C,
	breaklines=true,
	breakatwhitespace=true,
	postbreak=\hbox{$\hookrightarrow$ },
	showstringspaces=false,
	tabsize=4
}

\renewcommand*{\lstlistlistingname}{Listing catalog}

\renewcommand*{\appendixname}{Appendix}
\renewcommand*{\appendixtocname}{Appendices}
\renewcommand*{\appendixpagename}{Appendices}

\newcommand*{\mcite}[1]{\footnote{\cite{#1}}}


\begin{document}

\begin{titlepage}
	\begin{center}
		{\titlefont\huge Titel der Arbeit\par}

		\bigskip
		\bigskip

		{\titlefont\Large --- Projektbericht ---\par}

		\bigskip
		\bigskip

		{\large Arbeitsbereich Wissenschaftliches Rechnen\\
		Fachbereich Informatik\\
		Fakultät für Mathematik, Informatik und Naturwissenschaften\\
		Universität Hamburg\par}
	\end{center}

	\vfill

	{\large \begin{tabular}{ll}
		Vorgelegt von: & Lukas Stabe \\
			& Hans Ole Hatzel \\
			& Arne Struck \\
		E-Mail-Adresse: 
			& \href{mailto:2stabe@informatik.uni-hamburg.de} {2stabe@informatik.uni-hamburg.de} \\			
			& \href{mailto:2hatzel@informatik.uni-hamburg.de}{2hatzel@informatik.uni-hamburg.de} \\
			& \href{mailto:1struck@informatik.uni-hamburg.de}{1struck@informatik.uni-hamburg.de} \\ 
		%Matrikelnummer: & 1234567 \\
		Studiengang: & Informatik \\
		\\
		Betreuer: & Michael Kuhn \\
			& Konstantinos Chasapis \\
		\\
		Hamburg, den 29.03.2015
	\end{tabular}\par}
\end{titlepage}

\chapter*{Abstract}
%TODO write

\thispagestyle{empty}

   

\tableofcontents

\chapter{Introduction}
\label{Introduction}
%TODO ausbaubar
Since and even before the start of the days of modern computers compression is of a certain importance. 
One example for compression before modern computers are commercial codes for telegraphing.
\\
The use case in these days was to minimize the package size for network communication.
Many possible utilizations for compression have been discovered since then.
Among others fast, light compression for running systems and slower compression with higher compression ratios for long term storing. 
\\
Especially but not only in High Performance Computing (HPC) we see an emerging gap between the possibilities to create data and the ability to store it in an appropriate way. 
The gap is based on the reducing difference between computing speed and storage capacity\mcite{ExaStoSy}.
This effect can be observed when looking at new generations of supercomputers. 
In the case of the dkrz the storage capacity may only increase by a factor of three while computing performance is expected to grow by a factor of twenty\mcite{ExaStoSy}.
Compression can not solve this problem, but it can reduce the impact of the gap. 
\\
However, compression is a superordinate concept.
There are many possible ways (algorithms) to achieve compression.
The arising question is for which use case which way is the adequate one.
There is an uncountable number of different ways to compress.
This means the optimal solution to this problem is not detectable.
But it is possible to get an approximation through testing as many algortihms as possible an comparing the results.
That is the main point of this analysis of compression algorithms in the HPC field.


\chapter{Assignment and Methodology}
\label{Assignment and Methodology}
%TODO ausbauen
\section*{Assignment}
The main assignment for this project is the performance analysis for different compression algorithms in the HPC field. 
\\
Therefore we have to analyse them for the different use cases of compression in the HPC field. 
The main use cases are the storage in the life system and the storage in the archive system.
\\
Furthermore we have to look at the decompression speed and the ways it effects our two cases. 
\\
Analysing requires tools to get information about data. There are existing tools for testing multiple compression algorithms on a single file. So one task is to find a way to test multiple files with multiple algorithms.


\section*{Methodology}
We approached this through writing a script around a tool for testing multiple compression algorithms so we could use it for multiple files and store the results in a database. 
\\
Multiple files were necessary to get an appropriate amount of measurements.
For this purpose different datasets from scientific sources were required.
\\
After selecting and getting suitable data sets and testing algorithms on them with our script we needed to analyse the different outputs.
To do so we created a metric which can be weighted, so it is possible to map the results to our use cases in an appropriate way. \\

\chapter{Design and Implementation}
\label{Design}
%TODO expand
We implemented a script to test compression algorithms on multiple datasets based on the open source tool fsbench. 
Fsbench is written in C++ and has the option to output the data it gathers, machine readable, as csv.
However the tool is not capable of batch processing multiple files. \\
This is where our script comes in, we intend to provide a simple interface to test entire directory structures and store them in easy to use formats.
We decided to go with sqlite as a data store since it combines the simplicity of normal files with some of the performance and usage benefits of databases. \\
Our script aims to speed up the benchmarking process by launching multiple instances of fsbench on the different cores of the system.
For this we decided to use the python process pool implementation as it provides a simple way to split up the workload.
Our multiprocessing model therefore just works by enqueuing all files that are to be benchmarked (represented by there filenames) into a queue.
Each of the worker process will benchmark the compression of this file as soon at it is done with the previous task by starting an fsbench instance.
After each finished instance of the benchmarks, meaning one run of fsbench, the main process will get the results and write them into a sqlite database.


We provide a secondary script to help visualize the results. 
The library pyplot to help with this visualization as it provides a rich interface for displaying diagrams, graphs and the like.
The visualization script aims to provide some capabilities to filter the data and get a good first impression of the gathered data and was also used to create the diagrams in this document.

Along the way we encountered a few issues, after first implementing the whole thing using pythons thread safe queue we noticed that we needed further abstraction in order to ensure a readable script. We used the python process pool to achieve this.

Fsbench in it's non interactive mode, meaning the csv mode that is used for machine parsable results, also does not supply results such as the compression ratio but instead provides the data to calculate thes yourself. We decided to store data like this in the database since operations such as averaging the compression ratio of many files would otherwise be very slow.
While views would have been an option we decided that some 20-30\% increase in result size would be more bearable than contently having slow querys while analyzing the data.

\chapter{Measuring and Processing}
\label{Measuring and Processing}
%TODO continue
\section{Data Sources}
The examined data sets come from 5 sources and contain different kinds of information.
The first source is the KASCADE Cosmic Ray Data Centre (KCDC)\footnote{\url{https://kcdc.ikp.kit.edu/}} of the Karlsruhe Institute of Technology which provides data about cosmic rays. 
The data is stored as text in ASCII encoding. 
It's a table represented with space delimited numbers. Only small parts of the ASCII range are actually used for most of the file. Only at the beginning of the file there are headlines for each column, consisting of text, the rest of the files consists of mostly numbers.
Therefore the file is expected to be very compressible.
\\

The second source provides climatic data and is the Cera database from the DKRZ\footnote{\url{http://cera-www.dkrz.de/CERA/}}. 
It is stored as in the NetCDF format.
NetCDF is a binary format for storing array-like data that contains meta data describing the stored data.
%TODO what about internal compression of netcdf 
%TODO source: https://www.unidata.ucar.edu/software/netcdf/docs/netcdf.html#Introduction
\\

The third set of data sets is provided by the National Climatic Data Center (NCDC) and their global forecast system\footnote{\url{http://www.ncdc.noaa.gov/data-access/model-data/model-datasets/global-forcast-system-gfs}}. 
We used data weather data provided in the grb2 format with inv files for metadata. 
\\

The fourth source is the CERN\footnote{\url{http://opendata.cern.ch/}} which provides data from the Large Hadron Collider (LHC) in the root format (for their data analysis software).
ROOT files aim to be very compressed binary files. Therefore little compression is to be expected, but many files in HPC may be precompressed, therefore performance on compressed data could also be a very interesting metric.
% TODO source here: https://root.cern.ch/drupal/content/root-files-1
\\

Our final data set is the Silesia compression Corpus\footnote{\url{http://sun.aei.polsl.pl/~sdeor/index.php?page=silesia}}, a standard data set for testing lossless compression algorithms, therefor it contains a mutitude of data and formats. We used this to gain an initial impression of the algorithms and which aspects they excel in. While this dataset aims to be as diverse as possible, some of the datasets are also typical for HPC.
\\

\newpage
\section*{Block sizes}
Compression algorithms in HPC would ideally work in parallel as otherwise resources of HPC-systems are not used to their full extend.
A common approach for achieving parallelism in compression is by splitting up the data and compressing each the blocks individually on a different core or node\mcite{BenComp}.
%TODO Sauce on dis.
While we won't explore the implementations for splitting up the data and distributing it we can access the performance of different algorithms on smaller blocks of data to simulate their usage in parallel implementations.
By default our tools test with a block size of 1.7 GB, meaning that the implementation of each algorithm gets 1.7GB of data at a time. 
Some algorithms may split up this data on their own, like lz4 with it's sliding window. %TODO source for lz4
To simulate realistic splitting of these files for multiprocessing we decided to go with blocksizes of 500MB, 100MB, 10MB and 1MB.

%TODO put data here

As our data show, the impact of smaller block sizes is limited. Even for %TODO


\section{Results}
The first approach is to test algorithms on the Silesia Corpus to decide which algorithms will be looked at in more detail.
\\
Testing on the Silesia Compression Corpus leads to the results in figure \ref{fig:sc_res}. 
The corresponding values are shown in table \ref{tab:sc_res}. \\


%Data from benchmark_26031700.db, filtered with: %silesia%
\begin{figure}[h]
\begin{center}
	\scalebox{0.75}{%% Creator: Matplotlib, PGF backend
%%
%% To include the figure in your LaTeX document, write
%%   \input{<filename>.pgf}
%%
%% Make sure the required packages are loaded in your preamble
%%   \usepackage{pgf}
%%
%% Figures using additional raster images can only be included by \input if
%% they are in the same directory as the main LaTeX file. For loading figures
%% from other directories you can use the `import` package
%%   \usepackage{import}
%% and then include the figures with
%%   \import{<path to file>}{<filename>.pgf}
%%
%% Matplotlib used the following preamble
%%   \usepackage{fontspec}
%%   \setmainfont{DejaVu Serif}
%%   \setsansfont{DejaVu Sans}
%%   \setmonofont{DejaVu Sans Mono}
%%
\begingroup%
\makeatletter%
\begin{pgfpicture}%
\pgfpathrectangle{\pgfpointorigin}{\pgfqpoint{8.000000in}{6.000000in}}%
\pgfusepath{use as bounding box, clip}%
\begin{pgfscope}%
\pgfsetbuttcap%
\pgfsetmiterjoin%
\definecolor{currentfill}{rgb}{1.000000,1.000000,1.000000}%
\pgfsetfillcolor{currentfill}%
\pgfsetlinewidth{0.000000pt}%
\definecolor{currentstroke}{rgb}{1.000000,1.000000,1.000000}%
\pgfsetstrokecolor{currentstroke}%
\pgfsetdash{}{0pt}%
\pgfpathmoveto{\pgfqpoint{0.000000in}{0.000000in}}%
\pgfpathlineto{\pgfqpoint{8.000000in}{0.000000in}}%
\pgfpathlineto{\pgfqpoint{8.000000in}{6.000000in}}%
\pgfpathlineto{\pgfqpoint{0.000000in}{6.000000in}}%
\pgfpathclose%
\pgfusepath{fill}%
\end{pgfscope}%
\begin{pgfscope}%
\pgfsetbuttcap%
\pgfsetmiterjoin%
\definecolor{currentfill}{rgb}{1.000000,1.000000,1.000000}%
\pgfsetfillcolor{currentfill}%
\pgfsetlinewidth{0.000000pt}%
\definecolor{currentstroke}{rgb}{0.000000,0.000000,0.000000}%
\pgfsetstrokecolor{currentstroke}%
\pgfsetstrokeopacity{0.000000}%
\pgfsetdash{}{0pt}%
\pgfpathmoveto{\pgfqpoint{1.000000in}{0.600000in}}%
\pgfpathlineto{\pgfqpoint{7.200000in}{0.600000in}}%
\pgfpathlineto{\pgfqpoint{7.200000in}{5.400000in}}%
\pgfpathlineto{\pgfqpoint{1.000000in}{5.400000in}}%
\pgfpathclose%
\pgfusepath{fill}%
\end{pgfscope}%
\begin{pgfscope}%
\pgfpathrectangle{\pgfqpoint{1.000000in}{0.600000in}}{\pgfqpoint{6.200000in}{4.800000in}} %
\pgfusepath{clip}%
\pgfsetbuttcap%
\pgfsetroundjoin%
\definecolor{currentfill}{rgb}{0.000000,0.000000,1.000000}%
\pgfsetfillcolor{currentfill}%
\pgfsetlinewidth{1.003750pt}%
\definecolor{currentstroke}{rgb}{0.000000,0.000000,1.000000}%
\pgfsetstrokecolor{currentstroke}%
\pgfsetdash{}{0pt}%
\pgfpathmoveto{\pgfqpoint{4.622909in}{0.629206in}}%
\pgfpathlineto{\pgfqpoint{4.685022in}{0.691319in}}%
\pgfpathmoveto{\pgfqpoint{4.622909in}{0.691319in}}%
\pgfpathlineto{\pgfqpoint{4.685022in}{0.629206in}}%
\pgfusepath{stroke,fill}%
\end{pgfscope}%
\begin{pgfscope}%
\pgfpathrectangle{\pgfqpoint{1.000000in}{0.600000in}}{\pgfqpoint{6.200000in}{4.800000in}} %
\pgfusepath{clip}%
\pgfsetbuttcap%
\pgfsetroundjoin%
\definecolor{currentfill}{rgb}{0.000000,0.000000,1.000000}%
\pgfsetfillcolor{currentfill}%
\pgfsetlinewidth{1.003750pt}%
\definecolor{currentstroke}{rgb}{0.000000,0.000000,1.000000}%
\pgfsetstrokecolor{currentstroke}%
\pgfsetdash{}{0pt}%
\pgfpathmoveto{\pgfqpoint{4.834271in}{0.623089in}}%
\pgfpathlineto{\pgfqpoint{4.896384in}{0.685202in}}%
\pgfpathmoveto{\pgfqpoint{4.834271in}{0.685202in}}%
\pgfpathlineto{\pgfqpoint{4.896384in}{0.623089in}}%
\pgfusepath{stroke,fill}%
\end{pgfscope}%
\begin{pgfscope}%
\pgfpathrectangle{\pgfqpoint{1.000000in}{0.600000in}}{\pgfqpoint{6.200000in}{4.800000in}} %
\pgfusepath{clip}%
\pgfsetbuttcap%
\pgfsetroundjoin%
\definecolor{currentfill}{rgb}{0.000000,0.000000,1.000000}%
\pgfsetfillcolor{currentfill}%
\pgfsetlinewidth{1.003750pt}%
\definecolor{currentstroke}{rgb}{0.000000,0.000000,1.000000}%
\pgfsetstrokecolor{currentstroke}%
\pgfsetdash{}{0pt}%
\pgfpathmoveto{\pgfqpoint{6.884127in}{0.631942in}}%
\pgfpathlineto{\pgfqpoint{6.946240in}{0.694055in}}%
\pgfpathmoveto{\pgfqpoint{6.884127in}{0.694055in}}%
\pgfpathlineto{\pgfqpoint{6.946240in}{0.631942in}}%
\pgfusepath{stroke,fill}%
\end{pgfscope}%
\begin{pgfscope}%
\pgfpathrectangle{\pgfqpoint{1.000000in}{0.600000in}}{\pgfqpoint{6.200000in}{4.800000in}} %
\pgfusepath{clip}%
\pgfsetbuttcap%
\pgfsetroundjoin%
\definecolor{currentfill}{rgb}{0.000000,0.000000,1.000000}%
\pgfsetfillcolor{currentfill}%
\pgfsetlinewidth{1.003750pt}%
\definecolor{currentstroke}{rgb}{0.000000,0.000000,1.000000}%
\pgfsetstrokecolor{currentstroke}%
\pgfsetdash{}{0pt}%
\pgfpathmoveto{\pgfqpoint{2.131178in}{3.343934in}}%
\pgfpathlineto{\pgfqpoint{2.193291in}{3.406047in}}%
\pgfpathmoveto{\pgfqpoint{2.131178in}{3.406047in}}%
\pgfpathlineto{\pgfqpoint{2.193291in}{3.343934in}}%
\pgfusepath{stroke,fill}%
\end{pgfscope}%
\begin{pgfscope}%
\pgfpathrectangle{\pgfqpoint{1.000000in}{0.600000in}}{\pgfqpoint{6.200000in}{4.800000in}} %
\pgfusepath{clip}%
\pgfsetbuttcap%
\pgfsetroundjoin%
\definecolor{currentfill}{rgb}{0.000000,0.000000,1.000000}%
\pgfsetfillcolor{currentfill}%
\pgfsetlinewidth{1.003750pt}%
\definecolor{currentstroke}{rgb}{0.000000,0.000000,1.000000}%
\pgfsetstrokecolor{currentstroke}%
\pgfsetdash{}{0pt}%
\pgfpathmoveto{\pgfqpoint{1.353507in}{1.945044in}}%
\pgfpathlineto{\pgfqpoint{1.415620in}{2.007157in}}%
\pgfpathmoveto{\pgfqpoint{1.353507in}{2.007157in}}%
\pgfpathlineto{\pgfqpoint{1.415620in}{1.945044in}}%
\pgfusepath{stroke,fill}%
\end{pgfscope}%
\begin{pgfscope}%
\pgfpathrectangle{\pgfqpoint{1.000000in}{0.600000in}}{\pgfqpoint{6.200000in}{4.800000in}} %
\pgfusepath{clip}%
\pgfsetbuttcap%
\pgfsetroundjoin%
\definecolor{currentfill}{rgb}{0.000000,0.000000,1.000000}%
\pgfsetfillcolor{currentfill}%
\pgfsetlinewidth{1.003750pt}%
\definecolor{currentstroke}{rgb}{0.000000,0.000000,1.000000}%
\pgfsetstrokecolor{currentstroke}%
\pgfsetdash{}{0pt}%
\pgfpathmoveto{\pgfqpoint{3.666954in}{0.783752in}}%
\pgfpathlineto{\pgfqpoint{3.729067in}{0.845865in}}%
\pgfpathmoveto{\pgfqpoint{3.666954in}{0.845865in}}%
\pgfpathlineto{\pgfqpoint{3.729067in}{0.783752in}}%
\pgfusepath{stroke,fill}%
\end{pgfscope}%
\begin{pgfscope}%
\pgfpathrectangle{\pgfqpoint{1.000000in}{0.600000in}}{\pgfqpoint{6.200000in}{4.800000in}} %
\pgfusepath{clip}%
\pgfsetbuttcap%
\pgfsetroundjoin%
\definecolor{currentfill}{rgb}{0.000000,0.000000,1.000000}%
\pgfsetfillcolor{currentfill}%
\pgfsetlinewidth{1.003750pt}%
\definecolor{currentstroke}{rgb}{0.000000,0.000000,1.000000}%
\pgfsetstrokecolor{currentstroke}%
\pgfsetdash{}{0pt}%
\pgfpathmoveto{\pgfqpoint{2.119281in}{5.074270in}}%
\pgfpathlineto{\pgfqpoint{2.181394in}{5.136383in}}%
\pgfpathmoveto{\pgfqpoint{2.119281in}{5.136383in}}%
\pgfpathlineto{\pgfqpoint{2.181394in}{5.074270in}}%
\pgfusepath{stroke,fill}%
\end{pgfscope}%
\begin{pgfscope}%
\pgfpathrectangle{\pgfqpoint{1.000000in}{0.600000in}}{\pgfqpoint{6.200000in}{4.800000in}} %
\pgfusepath{clip}%
\pgfsetbuttcap%
\pgfsetroundjoin%
\definecolor{currentfill}{rgb}{0.000000,0.000000,1.000000}%
\pgfsetfillcolor{currentfill}%
\pgfsetlinewidth{1.003750pt}%
\definecolor{currentstroke}{rgb}{0.000000,0.000000,1.000000}%
\pgfsetstrokecolor{currentstroke}%
\pgfsetdash{}{0pt}%
\pgfpathmoveto{\pgfqpoint{4.354091in}{0.737279in}}%
\pgfpathlineto{\pgfqpoint{4.416204in}{0.799392in}}%
\pgfpathmoveto{\pgfqpoint{4.354091in}{0.799392in}}%
\pgfpathlineto{\pgfqpoint{4.416204in}{0.737279in}}%
\pgfusepath{stroke,fill}%
\end{pgfscope}%
\begin{pgfscope}%
\pgfpathrectangle{\pgfqpoint{1.000000in}{0.600000in}}{\pgfqpoint{6.200000in}{4.800000in}} %
\pgfusepath{clip}%
\pgfsetbuttcap%
\pgfsetroundjoin%
\definecolor{currentfill}{rgb}{0.000000,0.000000,1.000000}%
\pgfsetfillcolor{currentfill}%
\pgfsetlinewidth{1.003750pt}%
\definecolor{currentstroke}{rgb}{0.000000,0.000000,1.000000}%
\pgfsetstrokecolor{currentstroke}%
\pgfsetdash{}{0pt}%
\pgfpathmoveto{\pgfqpoint{3.930967in}{1.940433in}}%
\pgfpathlineto{\pgfqpoint{3.993080in}{2.002546in}}%
\pgfpathmoveto{\pgfqpoint{3.930967in}{2.002546in}}%
\pgfpathlineto{\pgfqpoint{3.993080in}{1.940433in}}%
\pgfusepath{stroke,fill}%
\end{pgfscope}%
\begin{pgfscope}%
\pgfsetrectcap%
\pgfsetmiterjoin%
\pgfsetlinewidth{1.003750pt}%
\definecolor{currentstroke}{rgb}{0.000000,0.000000,0.000000}%
\pgfsetstrokecolor{currentstroke}%
\pgfsetdash{}{0pt}%
\pgfpathmoveto{\pgfqpoint{1.000000in}{0.600000in}}%
\pgfpathlineto{\pgfqpoint{7.200000in}{0.600000in}}%
\pgfusepath{stroke}%
\end{pgfscope}%
\begin{pgfscope}%
\pgfsetrectcap%
\pgfsetmiterjoin%
\pgfsetlinewidth{1.003750pt}%
\definecolor{currentstroke}{rgb}{0.000000,0.000000,0.000000}%
\pgfsetstrokecolor{currentstroke}%
\pgfsetdash{}{0pt}%
\pgfpathmoveto{\pgfqpoint{1.000000in}{0.600000in}}%
\pgfpathlineto{\pgfqpoint{1.000000in}{5.400000in}}%
\pgfusepath{stroke}%
\end{pgfscope}%
\begin{pgfscope}%
\pgfsetrectcap%
\pgfsetmiterjoin%
\pgfsetlinewidth{1.003750pt}%
\definecolor{currentstroke}{rgb}{0.000000,0.000000,0.000000}%
\pgfsetstrokecolor{currentstroke}%
\pgfsetdash{}{0pt}%
\pgfpathmoveto{\pgfqpoint{1.000000in}{5.400000in}}%
\pgfpathlineto{\pgfqpoint{7.200000in}{5.400000in}}%
\pgfusepath{stroke}%
\end{pgfscope}%
\begin{pgfscope}%
\pgfsetrectcap%
\pgfsetmiterjoin%
\pgfsetlinewidth{1.003750pt}%
\definecolor{currentstroke}{rgb}{0.000000,0.000000,0.000000}%
\pgfsetstrokecolor{currentstroke}%
\pgfsetdash{}{0pt}%
\pgfpathmoveto{\pgfqpoint{7.200000in}{0.600000in}}%
\pgfpathlineto{\pgfqpoint{7.200000in}{5.400000in}}%
\pgfusepath{stroke}%
\end{pgfscope}%
\begin{pgfscope}%
\pgfsetbuttcap%
\pgfsetroundjoin%
\definecolor{currentfill}{rgb}{0.000000,0.000000,0.000000}%
\pgfsetfillcolor{currentfill}%
\pgfsetlinewidth{0.501875pt}%
\definecolor{currentstroke}{rgb}{0.000000,0.000000,0.000000}%
\pgfsetstrokecolor{currentstroke}%
\pgfsetdash{}{0pt}%
\pgfsys@defobject{currentmarker}{\pgfqpoint{0.000000in}{0.000000in}}{\pgfqpoint{0.000000in}{0.055556in}}{%
\pgfpathmoveto{\pgfqpoint{0.000000in}{0.000000in}}%
\pgfpathlineto{\pgfqpoint{0.000000in}{0.055556in}}%
\pgfusepath{stroke,fill}%
}%
\begin{pgfscope}%
\pgfsys@transformshift{1.764040in}{0.600000in}%
\pgfsys@useobject{currentmarker}{}%
\end{pgfscope}%
\end{pgfscope}%
\begin{pgfscope}%
\pgfsetbuttcap%
\pgfsetroundjoin%
\definecolor{currentfill}{rgb}{0.000000,0.000000,0.000000}%
\pgfsetfillcolor{currentfill}%
\pgfsetlinewidth{0.501875pt}%
\definecolor{currentstroke}{rgb}{0.000000,0.000000,0.000000}%
\pgfsetstrokecolor{currentstroke}%
\pgfsetdash{}{0pt}%
\pgfsys@defobject{currentmarker}{\pgfqpoint{0.000000in}{-0.055556in}}{\pgfqpoint{0.000000in}{0.000000in}}{%
\pgfpathmoveto{\pgfqpoint{0.000000in}{0.000000in}}%
\pgfpathlineto{\pgfqpoint{0.000000in}{-0.055556in}}%
\pgfusepath{stroke,fill}%
}%
\begin{pgfscope}%
\pgfsys@transformshift{1.764040in}{5.400000in}%
\pgfsys@useobject{currentmarker}{}%
\end{pgfscope}%
\end{pgfscope}%
\begin{pgfscope}%
\pgftext[x=1.764040in,y=0.544444in,,top]{\sffamily\fontsize{12.000000}{14.400000}\selectfont 2.0}%
\end{pgfscope}%
\begin{pgfscope}%
\pgfsetbuttcap%
\pgfsetroundjoin%
\definecolor{currentfill}{rgb}{0.000000,0.000000,0.000000}%
\pgfsetfillcolor{currentfill}%
\pgfsetlinewidth{0.501875pt}%
\definecolor{currentstroke}{rgb}{0.000000,0.000000,0.000000}%
\pgfsetstrokecolor{currentstroke}%
\pgfsetdash{}{0pt}%
\pgfsys@defobject{currentmarker}{\pgfqpoint{0.000000in}{0.000000in}}{\pgfqpoint{0.000000in}{0.055556in}}{%
\pgfpathmoveto{\pgfqpoint{0.000000in}{0.000000in}}%
\pgfpathlineto{\pgfqpoint{0.000000in}{0.055556in}}%
\pgfusepath{stroke,fill}%
}%
\begin{pgfscope}%
\pgfsys@transformshift{2.540606in}{0.600000in}%
\pgfsys@useobject{currentmarker}{}%
\end{pgfscope}%
\end{pgfscope}%
\begin{pgfscope}%
\pgfsetbuttcap%
\pgfsetroundjoin%
\definecolor{currentfill}{rgb}{0.000000,0.000000,0.000000}%
\pgfsetfillcolor{currentfill}%
\pgfsetlinewidth{0.501875pt}%
\definecolor{currentstroke}{rgb}{0.000000,0.000000,0.000000}%
\pgfsetstrokecolor{currentstroke}%
\pgfsetdash{}{0pt}%
\pgfsys@defobject{currentmarker}{\pgfqpoint{0.000000in}{-0.055556in}}{\pgfqpoint{0.000000in}{0.000000in}}{%
\pgfpathmoveto{\pgfqpoint{0.000000in}{0.000000in}}%
\pgfpathlineto{\pgfqpoint{0.000000in}{-0.055556in}}%
\pgfusepath{stroke,fill}%
}%
\begin{pgfscope}%
\pgfsys@transformshift{2.540606in}{5.400000in}%
\pgfsys@useobject{currentmarker}{}%
\end{pgfscope}%
\end{pgfscope}%
\begin{pgfscope}%
\pgftext[x=2.540606in,y=0.544444in,,top]{\sffamily\fontsize{12.000000}{14.400000}\selectfont 2.5}%
\end{pgfscope}%
\begin{pgfscope}%
\pgfsetbuttcap%
\pgfsetroundjoin%
\definecolor{currentfill}{rgb}{0.000000,0.000000,0.000000}%
\pgfsetfillcolor{currentfill}%
\pgfsetlinewidth{0.501875pt}%
\definecolor{currentstroke}{rgb}{0.000000,0.000000,0.000000}%
\pgfsetstrokecolor{currentstroke}%
\pgfsetdash{}{0pt}%
\pgfsys@defobject{currentmarker}{\pgfqpoint{0.000000in}{0.000000in}}{\pgfqpoint{0.000000in}{0.055556in}}{%
\pgfpathmoveto{\pgfqpoint{0.000000in}{0.000000in}}%
\pgfpathlineto{\pgfqpoint{0.000000in}{0.055556in}}%
\pgfusepath{stroke,fill}%
}%
\begin{pgfscope}%
\pgfsys@transformshift{3.317172in}{0.600000in}%
\pgfsys@useobject{currentmarker}{}%
\end{pgfscope}%
\end{pgfscope}%
\begin{pgfscope}%
\pgfsetbuttcap%
\pgfsetroundjoin%
\definecolor{currentfill}{rgb}{0.000000,0.000000,0.000000}%
\pgfsetfillcolor{currentfill}%
\pgfsetlinewidth{0.501875pt}%
\definecolor{currentstroke}{rgb}{0.000000,0.000000,0.000000}%
\pgfsetstrokecolor{currentstroke}%
\pgfsetdash{}{0pt}%
\pgfsys@defobject{currentmarker}{\pgfqpoint{0.000000in}{-0.055556in}}{\pgfqpoint{0.000000in}{0.000000in}}{%
\pgfpathmoveto{\pgfqpoint{0.000000in}{0.000000in}}%
\pgfpathlineto{\pgfqpoint{0.000000in}{-0.055556in}}%
\pgfusepath{stroke,fill}%
}%
\begin{pgfscope}%
\pgfsys@transformshift{3.317172in}{5.400000in}%
\pgfsys@useobject{currentmarker}{}%
\end{pgfscope}%
\end{pgfscope}%
\begin{pgfscope}%
\pgftext[x=3.317172in,y=0.544444in,,top]{\sffamily\fontsize{12.000000}{14.400000}\selectfont 3.0}%
\end{pgfscope}%
\begin{pgfscope}%
\pgfsetbuttcap%
\pgfsetroundjoin%
\definecolor{currentfill}{rgb}{0.000000,0.000000,0.000000}%
\pgfsetfillcolor{currentfill}%
\pgfsetlinewidth{0.501875pt}%
\definecolor{currentstroke}{rgb}{0.000000,0.000000,0.000000}%
\pgfsetstrokecolor{currentstroke}%
\pgfsetdash{}{0pt}%
\pgfsys@defobject{currentmarker}{\pgfqpoint{0.000000in}{0.000000in}}{\pgfqpoint{0.000000in}{0.055556in}}{%
\pgfpathmoveto{\pgfqpoint{0.000000in}{0.000000in}}%
\pgfpathlineto{\pgfqpoint{0.000000in}{0.055556in}}%
\pgfusepath{stroke,fill}%
}%
\begin{pgfscope}%
\pgfsys@transformshift{4.093737in}{0.600000in}%
\pgfsys@useobject{currentmarker}{}%
\end{pgfscope}%
\end{pgfscope}%
\begin{pgfscope}%
\pgfsetbuttcap%
\pgfsetroundjoin%
\definecolor{currentfill}{rgb}{0.000000,0.000000,0.000000}%
\pgfsetfillcolor{currentfill}%
\pgfsetlinewidth{0.501875pt}%
\definecolor{currentstroke}{rgb}{0.000000,0.000000,0.000000}%
\pgfsetstrokecolor{currentstroke}%
\pgfsetdash{}{0pt}%
\pgfsys@defobject{currentmarker}{\pgfqpoint{0.000000in}{-0.055556in}}{\pgfqpoint{0.000000in}{0.000000in}}{%
\pgfpathmoveto{\pgfqpoint{0.000000in}{0.000000in}}%
\pgfpathlineto{\pgfqpoint{0.000000in}{-0.055556in}}%
\pgfusepath{stroke,fill}%
}%
\begin{pgfscope}%
\pgfsys@transformshift{4.093737in}{5.400000in}%
\pgfsys@useobject{currentmarker}{}%
\end{pgfscope}%
\end{pgfscope}%
\begin{pgfscope}%
\pgftext[x=4.093737in,y=0.544444in,,top]{\sffamily\fontsize{12.000000}{14.400000}\selectfont 3.5}%
\end{pgfscope}%
\begin{pgfscope}%
\pgfsetbuttcap%
\pgfsetroundjoin%
\definecolor{currentfill}{rgb}{0.000000,0.000000,0.000000}%
\pgfsetfillcolor{currentfill}%
\pgfsetlinewidth{0.501875pt}%
\definecolor{currentstroke}{rgb}{0.000000,0.000000,0.000000}%
\pgfsetstrokecolor{currentstroke}%
\pgfsetdash{}{0pt}%
\pgfsys@defobject{currentmarker}{\pgfqpoint{0.000000in}{0.000000in}}{\pgfqpoint{0.000000in}{0.055556in}}{%
\pgfpathmoveto{\pgfqpoint{0.000000in}{0.000000in}}%
\pgfpathlineto{\pgfqpoint{0.000000in}{0.055556in}}%
\pgfusepath{stroke,fill}%
}%
\begin{pgfscope}%
\pgfsys@transformshift{4.870303in}{0.600000in}%
\pgfsys@useobject{currentmarker}{}%
\end{pgfscope}%
\end{pgfscope}%
\begin{pgfscope}%
\pgfsetbuttcap%
\pgfsetroundjoin%
\definecolor{currentfill}{rgb}{0.000000,0.000000,0.000000}%
\pgfsetfillcolor{currentfill}%
\pgfsetlinewidth{0.501875pt}%
\definecolor{currentstroke}{rgb}{0.000000,0.000000,0.000000}%
\pgfsetstrokecolor{currentstroke}%
\pgfsetdash{}{0pt}%
\pgfsys@defobject{currentmarker}{\pgfqpoint{0.000000in}{-0.055556in}}{\pgfqpoint{0.000000in}{0.000000in}}{%
\pgfpathmoveto{\pgfqpoint{0.000000in}{0.000000in}}%
\pgfpathlineto{\pgfqpoint{0.000000in}{-0.055556in}}%
\pgfusepath{stroke,fill}%
}%
\begin{pgfscope}%
\pgfsys@transformshift{4.870303in}{5.400000in}%
\pgfsys@useobject{currentmarker}{}%
\end{pgfscope}%
\end{pgfscope}%
\begin{pgfscope}%
\pgftext[x=4.870303in,y=0.544444in,,top]{\sffamily\fontsize{12.000000}{14.400000}\selectfont 4.0}%
\end{pgfscope}%
\begin{pgfscope}%
\pgfsetbuttcap%
\pgfsetroundjoin%
\definecolor{currentfill}{rgb}{0.000000,0.000000,0.000000}%
\pgfsetfillcolor{currentfill}%
\pgfsetlinewidth{0.501875pt}%
\definecolor{currentstroke}{rgb}{0.000000,0.000000,0.000000}%
\pgfsetstrokecolor{currentstroke}%
\pgfsetdash{}{0pt}%
\pgfsys@defobject{currentmarker}{\pgfqpoint{0.000000in}{0.000000in}}{\pgfqpoint{0.000000in}{0.055556in}}{%
\pgfpathmoveto{\pgfqpoint{0.000000in}{0.000000in}}%
\pgfpathlineto{\pgfqpoint{0.000000in}{0.055556in}}%
\pgfusepath{stroke,fill}%
}%
\begin{pgfscope}%
\pgfsys@transformshift{5.646869in}{0.600000in}%
\pgfsys@useobject{currentmarker}{}%
\end{pgfscope}%
\end{pgfscope}%
\begin{pgfscope}%
\pgfsetbuttcap%
\pgfsetroundjoin%
\definecolor{currentfill}{rgb}{0.000000,0.000000,0.000000}%
\pgfsetfillcolor{currentfill}%
\pgfsetlinewidth{0.501875pt}%
\definecolor{currentstroke}{rgb}{0.000000,0.000000,0.000000}%
\pgfsetstrokecolor{currentstroke}%
\pgfsetdash{}{0pt}%
\pgfsys@defobject{currentmarker}{\pgfqpoint{0.000000in}{-0.055556in}}{\pgfqpoint{0.000000in}{0.000000in}}{%
\pgfpathmoveto{\pgfqpoint{0.000000in}{0.000000in}}%
\pgfpathlineto{\pgfqpoint{0.000000in}{-0.055556in}}%
\pgfusepath{stroke,fill}%
}%
\begin{pgfscope}%
\pgfsys@transformshift{5.646869in}{5.400000in}%
\pgfsys@useobject{currentmarker}{}%
\end{pgfscope}%
\end{pgfscope}%
\begin{pgfscope}%
\pgftext[x=5.646869in,y=0.544444in,,top]{\sffamily\fontsize{12.000000}{14.400000}\selectfont 4.5}%
\end{pgfscope}%
\begin{pgfscope}%
\pgfsetbuttcap%
\pgfsetroundjoin%
\definecolor{currentfill}{rgb}{0.000000,0.000000,0.000000}%
\pgfsetfillcolor{currentfill}%
\pgfsetlinewidth{0.501875pt}%
\definecolor{currentstroke}{rgb}{0.000000,0.000000,0.000000}%
\pgfsetstrokecolor{currentstroke}%
\pgfsetdash{}{0pt}%
\pgfsys@defobject{currentmarker}{\pgfqpoint{0.000000in}{0.000000in}}{\pgfqpoint{0.000000in}{0.055556in}}{%
\pgfpathmoveto{\pgfqpoint{0.000000in}{0.000000in}}%
\pgfpathlineto{\pgfqpoint{0.000000in}{0.055556in}}%
\pgfusepath{stroke,fill}%
}%
\begin{pgfscope}%
\pgfsys@transformshift{6.423434in}{0.600000in}%
\pgfsys@useobject{currentmarker}{}%
\end{pgfscope}%
\end{pgfscope}%
\begin{pgfscope}%
\pgfsetbuttcap%
\pgfsetroundjoin%
\definecolor{currentfill}{rgb}{0.000000,0.000000,0.000000}%
\pgfsetfillcolor{currentfill}%
\pgfsetlinewidth{0.501875pt}%
\definecolor{currentstroke}{rgb}{0.000000,0.000000,0.000000}%
\pgfsetstrokecolor{currentstroke}%
\pgfsetdash{}{0pt}%
\pgfsys@defobject{currentmarker}{\pgfqpoint{0.000000in}{-0.055556in}}{\pgfqpoint{0.000000in}{0.000000in}}{%
\pgfpathmoveto{\pgfqpoint{0.000000in}{0.000000in}}%
\pgfpathlineto{\pgfqpoint{0.000000in}{-0.055556in}}%
\pgfusepath{stroke,fill}%
}%
\begin{pgfscope}%
\pgfsys@transformshift{6.423434in}{5.400000in}%
\pgfsys@useobject{currentmarker}{}%
\end{pgfscope}%
\end{pgfscope}%
\begin{pgfscope}%
\pgftext[x=6.423434in,y=0.544444in,,top]{\sffamily\fontsize{12.000000}{14.400000}\selectfont 5.0}%
\end{pgfscope}%
\begin{pgfscope}%
\pgfsetbuttcap%
\pgfsetroundjoin%
\definecolor{currentfill}{rgb}{0.000000,0.000000,0.000000}%
\pgfsetfillcolor{currentfill}%
\pgfsetlinewidth{0.501875pt}%
\definecolor{currentstroke}{rgb}{0.000000,0.000000,0.000000}%
\pgfsetstrokecolor{currentstroke}%
\pgfsetdash{}{0pt}%
\pgfsys@defobject{currentmarker}{\pgfqpoint{0.000000in}{0.000000in}}{\pgfqpoint{0.000000in}{0.055556in}}{%
\pgfpathmoveto{\pgfqpoint{0.000000in}{0.000000in}}%
\pgfpathlineto{\pgfqpoint{0.000000in}{0.055556in}}%
\pgfusepath{stroke,fill}%
}%
\begin{pgfscope}%
\pgfsys@transformshift{7.200000in}{0.600000in}%
\pgfsys@useobject{currentmarker}{}%
\end{pgfscope}%
\end{pgfscope}%
\begin{pgfscope}%
\pgfsetbuttcap%
\pgfsetroundjoin%
\definecolor{currentfill}{rgb}{0.000000,0.000000,0.000000}%
\pgfsetfillcolor{currentfill}%
\pgfsetlinewidth{0.501875pt}%
\definecolor{currentstroke}{rgb}{0.000000,0.000000,0.000000}%
\pgfsetstrokecolor{currentstroke}%
\pgfsetdash{}{0pt}%
\pgfsys@defobject{currentmarker}{\pgfqpoint{0.000000in}{-0.055556in}}{\pgfqpoint{0.000000in}{0.000000in}}{%
\pgfpathmoveto{\pgfqpoint{0.000000in}{0.000000in}}%
\pgfpathlineto{\pgfqpoint{0.000000in}{-0.055556in}}%
\pgfusepath{stroke,fill}%
}%
\begin{pgfscope}%
\pgfsys@transformshift{7.200000in}{5.400000in}%
\pgfsys@useobject{currentmarker}{}%
\end{pgfscope}%
\end{pgfscope}%
\begin{pgfscope}%
\pgftext[x=7.200000in,y=0.544444in,,top]{\sffamily\fontsize{12.000000}{14.400000}\selectfont 5.5}%
\end{pgfscope}%
\begin{pgfscope}%
\pgftext[x=4.100000in,y=0.313705in,,top]{\sffamily\fontsize{12.000000}{14.400000}\selectfont Compression Ratio}%
\end{pgfscope}%
\begin{pgfscope}%
\pgftext[x=0.944444in,y=0.597916in,right,]{\sffamily\fontsize{12.000000}{14.400000}\selectfont 0}%
\end{pgfscope}%
\begin{pgfscope}%
\pgfsetbuttcap%
\pgfsetroundjoin%
\definecolor{currentfill}{rgb}{0.000000,0.000000,0.000000}%
\pgfsetfillcolor{currentfill}%
\pgfsetlinewidth{0.501875pt}%
\definecolor{currentstroke}{rgb}{0.000000,0.000000,0.000000}%
\pgfsetstrokecolor{currentstroke}%
\pgfsetdash{}{0pt}%
\pgfsys@defobject{currentmarker}{\pgfqpoint{0.000000in}{0.000000in}}{\pgfqpoint{0.055556in}{0.000000in}}{%
\pgfpathmoveto{\pgfqpoint{0.000000in}{0.000000in}}%
\pgfpathlineto{\pgfqpoint{0.055556in}{0.000000in}}%
\pgfusepath{stroke,fill}%
}%
\begin{pgfscope}%
\pgfsys@transformshift{1.000000in}{1.398263in}%
\pgfsys@useobject{currentmarker}{}%
\end{pgfscope}%
\end{pgfscope}%
\begin{pgfscope}%
\pgfsetbuttcap%
\pgfsetroundjoin%
\definecolor{currentfill}{rgb}{0.000000,0.000000,0.000000}%
\pgfsetfillcolor{currentfill}%
\pgfsetlinewidth{0.501875pt}%
\definecolor{currentstroke}{rgb}{0.000000,0.000000,0.000000}%
\pgfsetstrokecolor{currentstroke}%
\pgfsetdash{}{0pt}%
\pgfsys@defobject{currentmarker}{\pgfqpoint{-0.055556in}{0.000000in}}{\pgfqpoint{0.000000in}{0.000000in}}{%
\pgfpathmoveto{\pgfqpoint{0.000000in}{0.000000in}}%
\pgfpathlineto{\pgfqpoint{-0.055556in}{0.000000in}}%
\pgfusepath{stroke,fill}%
}%
\begin{pgfscope}%
\pgfsys@transformshift{7.200000in}{1.398263in}%
\pgfsys@useobject{currentmarker}{}%
\end{pgfscope}%
\end{pgfscope}%
\begin{pgfscope}%
\pgftext[x=0.944444in,y=1.398263in,right,]{\sffamily\fontsize{12.000000}{14.400000}\selectfont 100}%
\end{pgfscope}%
\begin{pgfscope}%
\pgfsetbuttcap%
\pgfsetroundjoin%
\definecolor{currentfill}{rgb}{0.000000,0.000000,0.000000}%
\pgfsetfillcolor{currentfill}%
\pgfsetlinewidth{0.501875pt}%
\definecolor{currentstroke}{rgb}{0.000000,0.000000,0.000000}%
\pgfsetstrokecolor{currentstroke}%
\pgfsetdash{}{0pt}%
\pgfsys@defobject{currentmarker}{\pgfqpoint{0.000000in}{0.000000in}}{\pgfqpoint{0.055556in}{0.000000in}}{%
\pgfpathmoveto{\pgfqpoint{0.000000in}{0.000000in}}%
\pgfpathlineto{\pgfqpoint{0.055556in}{0.000000in}}%
\pgfusepath{stroke,fill}%
}%
\begin{pgfscope}%
\pgfsys@transformshift{1.000000in}{2.198611in}%
\pgfsys@useobject{currentmarker}{}%
\end{pgfscope}%
\end{pgfscope}%
\begin{pgfscope}%
\pgfsetbuttcap%
\pgfsetroundjoin%
\definecolor{currentfill}{rgb}{0.000000,0.000000,0.000000}%
\pgfsetfillcolor{currentfill}%
\pgfsetlinewidth{0.501875pt}%
\definecolor{currentstroke}{rgb}{0.000000,0.000000,0.000000}%
\pgfsetstrokecolor{currentstroke}%
\pgfsetdash{}{0pt}%
\pgfsys@defobject{currentmarker}{\pgfqpoint{-0.055556in}{0.000000in}}{\pgfqpoint{0.000000in}{0.000000in}}{%
\pgfpathmoveto{\pgfqpoint{0.000000in}{0.000000in}}%
\pgfpathlineto{\pgfqpoint{-0.055556in}{0.000000in}}%
\pgfusepath{stroke,fill}%
}%
\begin{pgfscope}%
\pgfsys@transformshift{7.200000in}{2.198611in}%
\pgfsys@useobject{currentmarker}{}%
\end{pgfscope}%
\end{pgfscope}%
\begin{pgfscope}%
\pgftext[x=0.944444in,y=2.198611in,right,]{\sffamily\fontsize{12.000000}{14.400000}\selectfont 200}%
\end{pgfscope}%
\begin{pgfscope}%
\pgfsetbuttcap%
\pgfsetroundjoin%
\definecolor{currentfill}{rgb}{0.000000,0.000000,0.000000}%
\pgfsetfillcolor{currentfill}%
\pgfsetlinewidth{0.501875pt}%
\definecolor{currentstroke}{rgb}{0.000000,0.000000,0.000000}%
\pgfsetstrokecolor{currentstroke}%
\pgfsetdash{}{0pt}%
\pgfsys@defobject{currentmarker}{\pgfqpoint{0.000000in}{0.000000in}}{\pgfqpoint{0.055556in}{0.000000in}}{%
\pgfpathmoveto{\pgfqpoint{0.000000in}{0.000000in}}%
\pgfpathlineto{\pgfqpoint{0.055556in}{0.000000in}}%
\pgfusepath{stroke,fill}%
}%
\begin{pgfscope}%
\pgfsys@transformshift{1.000000in}{2.998958in}%
\pgfsys@useobject{currentmarker}{}%
\end{pgfscope}%
\end{pgfscope}%
\begin{pgfscope}%
\pgfsetbuttcap%
\pgfsetroundjoin%
\definecolor{currentfill}{rgb}{0.000000,0.000000,0.000000}%
\pgfsetfillcolor{currentfill}%
\pgfsetlinewidth{0.501875pt}%
\definecolor{currentstroke}{rgb}{0.000000,0.000000,0.000000}%
\pgfsetstrokecolor{currentstroke}%
\pgfsetdash{}{0pt}%
\pgfsys@defobject{currentmarker}{\pgfqpoint{-0.055556in}{0.000000in}}{\pgfqpoint{0.000000in}{0.000000in}}{%
\pgfpathmoveto{\pgfqpoint{0.000000in}{0.000000in}}%
\pgfpathlineto{\pgfqpoint{-0.055556in}{0.000000in}}%
\pgfusepath{stroke,fill}%
}%
\begin{pgfscope}%
\pgfsys@transformshift{7.200000in}{2.998958in}%
\pgfsys@useobject{currentmarker}{}%
\end{pgfscope}%
\end{pgfscope}%
\begin{pgfscope}%
\pgftext[x=0.944444in,y=2.998958in,right,]{\sffamily\fontsize{12.000000}{14.400000}\selectfont 300}%
\end{pgfscope}%
\begin{pgfscope}%
\pgfsetbuttcap%
\pgfsetroundjoin%
\definecolor{currentfill}{rgb}{0.000000,0.000000,0.000000}%
\pgfsetfillcolor{currentfill}%
\pgfsetlinewidth{0.501875pt}%
\definecolor{currentstroke}{rgb}{0.000000,0.000000,0.000000}%
\pgfsetstrokecolor{currentstroke}%
\pgfsetdash{}{0pt}%
\pgfsys@defobject{currentmarker}{\pgfqpoint{0.000000in}{0.000000in}}{\pgfqpoint{0.055556in}{0.000000in}}{%
\pgfpathmoveto{\pgfqpoint{0.000000in}{0.000000in}}%
\pgfpathlineto{\pgfqpoint{0.055556in}{0.000000in}}%
\pgfusepath{stroke,fill}%
}%
\begin{pgfscope}%
\pgfsys@transformshift{1.000000in}{3.799305in}%
\pgfsys@useobject{currentmarker}{}%
\end{pgfscope}%
\end{pgfscope}%
\begin{pgfscope}%
\pgfsetbuttcap%
\pgfsetroundjoin%
\definecolor{currentfill}{rgb}{0.000000,0.000000,0.000000}%
\pgfsetfillcolor{currentfill}%
\pgfsetlinewidth{0.501875pt}%
\definecolor{currentstroke}{rgb}{0.000000,0.000000,0.000000}%
\pgfsetstrokecolor{currentstroke}%
\pgfsetdash{}{0pt}%
\pgfsys@defobject{currentmarker}{\pgfqpoint{-0.055556in}{0.000000in}}{\pgfqpoint{0.000000in}{0.000000in}}{%
\pgfpathmoveto{\pgfqpoint{0.000000in}{0.000000in}}%
\pgfpathlineto{\pgfqpoint{-0.055556in}{0.000000in}}%
\pgfusepath{stroke,fill}%
}%
\begin{pgfscope}%
\pgfsys@transformshift{7.200000in}{3.799305in}%
\pgfsys@useobject{currentmarker}{}%
\end{pgfscope}%
\end{pgfscope}%
\begin{pgfscope}%
\pgftext[x=0.944444in,y=3.799305in,right,]{\sffamily\fontsize{12.000000}{14.400000}\selectfont 400}%
\end{pgfscope}%
\begin{pgfscope}%
\pgfsetbuttcap%
\pgfsetroundjoin%
\definecolor{currentfill}{rgb}{0.000000,0.000000,0.000000}%
\pgfsetfillcolor{currentfill}%
\pgfsetlinewidth{0.501875pt}%
\definecolor{currentstroke}{rgb}{0.000000,0.000000,0.000000}%
\pgfsetstrokecolor{currentstroke}%
\pgfsetdash{}{0pt}%
\pgfsys@defobject{currentmarker}{\pgfqpoint{0.000000in}{0.000000in}}{\pgfqpoint{0.055556in}{0.000000in}}{%
\pgfpathmoveto{\pgfqpoint{0.000000in}{0.000000in}}%
\pgfpathlineto{\pgfqpoint{0.055556in}{0.000000in}}%
\pgfusepath{stroke,fill}%
}%
\begin{pgfscope}%
\pgfsys@transformshift{1.000000in}{4.599653in}%
\pgfsys@useobject{currentmarker}{}%
\end{pgfscope}%
\end{pgfscope}%
\begin{pgfscope}%
\pgfsetbuttcap%
\pgfsetroundjoin%
\definecolor{currentfill}{rgb}{0.000000,0.000000,0.000000}%
\pgfsetfillcolor{currentfill}%
\pgfsetlinewidth{0.501875pt}%
\definecolor{currentstroke}{rgb}{0.000000,0.000000,0.000000}%
\pgfsetstrokecolor{currentstroke}%
\pgfsetdash{}{0pt}%
\pgfsys@defobject{currentmarker}{\pgfqpoint{-0.055556in}{0.000000in}}{\pgfqpoint{0.000000in}{0.000000in}}{%
\pgfpathmoveto{\pgfqpoint{0.000000in}{0.000000in}}%
\pgfpathlineto{\pgfqpoint{-0.055556in}{0.000000in}}%
\pgfusepath{stroke,fill}%
}%
\begin{pgfscope}%
\pgfsys@transformshift{7.200000in}{4.599653in}%
\pgfsys@useobject{currentmarker}{}%
\end{pgfscope}%
\end{pgfscope}%
\begin{pgfscope}%
\pgftext[x=0.944444in,y=4.599653in,right,]{\sffamily\fontsize{12.000000}{14.400000}\selectfont 500}%
\end{pgfscope}%
\begin{pgfscope}%
\pgfsetbuttcap%
\pgfsetroundjoin%
\definecolor{currentfill}{rgb}{0.000000,0.000000,0.000000}%
\pgfsetfillcolor{currentfill}%
\pgfsetlinewidth{0.501875pt}%
\definecolor{currentstroke}{rgb}{0.000000,0.000000,0.000000}%
\pgfsetstrokecolor{currentstroke}%
\pgfsetdash{}{0pt}%
\pgfsys@defobject{currentmarker}{\pgfqpoint{0.000000in}{0.000000in}}{\pgfqpoint{0.055556in}{0.000000in}}{%
\pgfpathmoveto{\pgfqpoint{0.000000in}{0.000000in}}%
\pgfpathlineto{\pgfqpoint{0.055556in}{0.000000in}}%
\pgfusepath{stroke,fill}%
}%
\begin{pgfscope}%
\pgfsys@transformshift{1.000000in}{5.400000in}%
\pgfsys@useobject{currentmarker}{}%
\end{pgfscope}%
\end{pgfscope}%
\begin{pgfscope}%
\pgfsetbuttcap%
\pgfsetroundjoin%
\definecolor{currentfill}{rgb}{0.000000,0.000000,0.000000}%
\pgfsetfillcolor{currentfill}%
\pgfsetlinewidth{0.501875pt}%
\definecolor{currentstroke}{rgb}{0.000000,0.000000,0.000000}%
\pgfsetstrokecolor{currentstroke}%
\pgfsetdash{}{0pt}%
\pgfsys@defobject{currentmarker}{\pgfqpoint{-0.055556in}{0.000000in}}{\pgfqpoint{0.000000in}{0.000000in}}{%
\pgfpathmoveto{\pgfqpoint{0.000000in}{0.000000in}}%
\pgfpathlineto{\pgfqpoint{-0.055556in}{0.000000in}}%
\pgfusepath{stroke,fill}%
}%
\begin{pgfscope}%
\pgfsys@transformshift{7.200000in}{5.400000in}%
\pgfsys@useobject{currentmarker}{}%
\end{pgfscope}%
\end{pgfscope}%
\begin{pgfscope}%
\pgftext[x=0.944444in,y=5.400000in,right,]{\sffamily\fontsize{12.000000}{14.400000}\selectfont 600}%
\end{pgfscope}%
\begin{pgfscope}%
\pgftext[x=0.556885in,y=3.000000in,,bottom,rotate=90.000000]{\sffamily\fontsize{12.000000}{14.400000}\selectfont Compression speed MByte/s}%
\end{pgfscope}%
\begin{pgfscope}%
\pgftext[x=4.515077in,y=0.799152in,left,base]{\sffamily\fontsize{12.000000}{14.400000}\selectfont 7z-deflate}%
\end{pgfscope}%
\begin{pgfscope}%
\pgftext[x=4.906438in,y=0.675034in,left,base]{\sffamily\fontsize{12.000000}{14.400000}\selectfont 7z-deflate64}%
\end{pgfscope}%
\begin{pgfscope}%
\pgftext[x=6.776295in,y=0.801888in,left,base]{\sffamily\fontsize{12.000000}{14.400000}\selectfont bzip2}%
\end{pgfscope}%
\begin{pgfscope}%
\pgftext[x=2.023345in,y=3.513879in,left,base]{\sffamily\fontsize{12.000000}{14.400000}\selectfont LZ4}%
\end{pgfscope}%
\begin{pgfscope}%
\pgftext[x=1.245674in,y=2.114989in,left,base]{\sffamily\fontsize{12.000000}{14.400000}\selectfont lzjb}%
\end{pgfscope}%
\begin{pgfscope}%
\pgftext[x=3.559122in,y=0.953697in,left,base]{\sffamily\fontsize{12.000000}{14.400000}\selectfont lzmat}%
\end{pgfscope}%
\begin{pgfscope}%
\pgftext[x=2.011449in,y=5.244215in,left,base]{\sffamily\fontsize{12.000000}{14.400000}\selectfont LZO}%
\end{pgfscope}%
\begin{pgfscope}%
\pgftext[x=4.246259in,y=0.907224in,left,base]{\sffamily\fontsize{12.000000}{14.400000}\selectfont zlib}%
\end{pgfscope}%
\begin{pgfscope}%
\pgftext[x=3.823135in,y=2.110379in,left,base]{\sffamily\fontsize{12.000000}{14.400000}\selectfont ZSTD}%
\end{pgfscope}%
\end{pgfpicture}%
\makeatother%
\endgroup%
}
\end{center}
\caption{SC results plot}
\label{fig:sc_res}
\end{figure}

\quad \\
The graph looks at compression speed and compression ratio. A point to the top tight of another one means that, for the two metrics and our given dataset, the top right algorithms is flat out better. Given this dataset we decided to take a closer look at both lz4 and lzo since these are the algorithms that mostly focus on the speed and sacrifice compression ratio. We will take a look at zstd since it appears to make a uniquely balanced trade-off. \\
For the other extreme of the speed compression ratio trade-off we decided to go with bzip2. Both 7z-deflate variants are slower than bzip2 on the given dataset therfore we decided to go with zlib as well as an algorithm a little more balanced towards the speed aspect.

%Data from benchmark_26031700.db, filtered with: %silesia%
\begin{table}
\begin{center}
\scalebox{0.9}{
\begin{tabular}{|l|r|r|}
	\hline
	Algorithm & Compression Ratio & Compression speed MByte/s \\ \hline
	7z-deflate & 3.86 & 7.79 \\
	7z-deflate64 & 4.00 & 7.03 \\
	\textbf{bzip2} & 5.32 & 8.13 \\
	\textbf{LZ4} & 2.26 & 346.98 \\
	lzjb & 1.76 & 172.20 \\
	lzmat & 3.25 & 27.10 \\
	\textbf{LZO} & 2.25 & 563.18 \\
	\textbf{zlib} & 3.69 & 21.29 \\
	\textbf{ZSTD} & 3.42 & 171.62 \\ \hline
\end{tabular}
} 
\caption{SC results table, bold algorithms to be tested further}
\label{tab:sc_res}
\end{center}
\end{table}

\section{Evaluation}
In order to evaluate these results a comparison metric is required.
The resulting data can be interpreted as a set consiting of 2-tuples.
To get the values into the same value domain we have to normalize the values.
The base value for the normalization is the maximal value in its domain.
This results in:
\begin{center}
	\(
	   v(x) = \frac{v(x)}{max(v(x))}|\forall v \in \text{Results} \land \forall x \in \{ratio, speed\}
	\)
\end{center}

Since the Data consists of only positive components we can interprete the tuple as a vector and use the length of the normalized vectors as metric. 
Positive in this case connotes the greater the value the more efficient the algorithm.
At last we need to add the ability to weight the results. 
To achieve this we add a weighting vector whose elements will sum up to one and will be added as factors in the metric. 
This streches the unit circle to desired relations between compression speed and ratio.
For the normalized result vector \(V = (\text{compression ratio}, \text{speed})\) and the weighting vector \(W = (\text{compression ratio weighting}, \text{speed weighting})\) this can be expressed as:
\begin{center}
	\(
	\sum\limits^{1}_{i=0} \sqrt{(W(i) \cdot V(i))^2}
	\)
\end{center}
while
\begin{center}
	\(
	\sum\limits^{1}_{i=0}W(i) = 1
	\)
\end{center}
If we also want to adress decompression we can extend every vector with a third value for decompression speed.

	
\chapter{Discussion}
\label{Discussion}


\chapter{Conclusion}
\label{Conclusion}
%TODO write

\nocite{*}
\bibliographystyle{alpha}
\bibliography{literatur}

%TODO : an ref s denken!
\listoffigures

\listoftables

\lstlistoflistings

%TODO : was?
\newpage

\thispagestyle{empty}

\chapter*{}

%TODO needed?
\section*{Erklärung}

Wir versichern, dass wir die Arbeit selbstständig verfasst, die Arbeiten am Projekt selbstständig durchgeführt und keine anderen, als die angegebenen Hilfsmittel -- insbesondere keine im Quellenverzeichnis nicht benannten Internetquellen -- benutzt haben, die Arbeit vorher nicht in einem anderen Prüfungsverfahren eingereicht haben.

\smallskip

\bigskip
\bigskip
\bigskip

%Hamburg, den 29.03.2015  \quad \dotfill \\ \\
%%Data from 28_03_01_19.db, filtered with: None
\begin{tabular}{lrr}
\hline
 Unknown Metric   &   Compression Ratio &   Decompression speed MByte/s \\
\hline
 bzip2            &                4.17 &                         23.46 \\
 LZ4              &                2.00 &                     144626.61 \\
 LZO              &                2.01 &                     322814.64 \\
 zlib             &                3.09 &                        245.94 \\
 ZSTD             &                2.88 &                     287024.61 \\
\hline
\end{tabular}
%%Data from 28_03_01_19.db, filtered with: None
\begin{tabular}{lrr}
\hline
 Unknown Metric   &   Compression Ratio &   Compression speed MByte/s \\
\hline
 bzip2            &                4.17 &                        7.40 \\
 LZ4              &                2.00 &                      514.48 \\
 LZO              &                2.01 &                      828.04 \\
 zlib             &                3.09 &                       18.95 \\
 ZSTD             &                2.88 &                      238.66 \\
\hline
\end{tabular}
\end{document}

